\section{Sistemi lineari}
Un sistema lineare è un insieme di espressioni che hanno questa forma:
\[
    \begin{cases}
        a_{11}x_1 + \cdots + a_{1n}x_n = b_1 \\
        \cdots + \cdots + \cdots = \cdots \\
        a_{m1}x_1 + \cdots + a_{mn}a_n = b_m
    \end{cases}
\]
Dove:
\begin{itemize}
    \item $a_{ij}, b_i \in K$
    \item $x_n$ sono incognite
    \item $1 \leq i \leq m, 1 \leq j \leq n$
\end{itemize}

\hl{Un sistema lineare può anche essere scritto come un'equazione matriciale:}
\begin{align*}
    & A =
    \begin{bmatrix}
        a_{11} & \cdots & a_{1n} \\
        \vdots & \ddots & \vdots \\
        a_{m1} & \cdots & a_{mn}
    \end{bmatrix}, B =
    \begin{bmatrix}
        b_1 \\ \vdots \\ b_m
    \end{bmatrix}, X =
    \begin{bmatrix}
        x_1 \\ \vdots \\ x_n
    \end{bmatrix} \\
    & AX = B
\end{align*}

\hl{La matrice $[A|B]$ è detta la matrice completa del sistema. $[A|0]$ è detta
matrice del sistema omogeneo associato.}

\subsection{Sistema lineare omogeneo}
Un sistema lineare omogeneo è un \hl{sistema lineare del tipo $[A|0_{m1}]$} e avrà
sempre \hl{almeno una soluzione, di cui una $X=0_{n1}$}. La soluzione del sistema
lineare omogeneo \hl{è detta soluzione generale}. Un \hl{sistema lineare omogeneo e
il corrispettivo sistema normale condividono la soluzione generale}:
\begin{align*}
    \begin{cases}
        x + y + z = 0 \\
        z = 0
    \end{cases} & X_0 =
    \begin{bmatrix}
        -t \\
        t \\
        0
    \end{bmatrix} \\
    \begin{cases}
        x + y + z = 1 \\
        z = 0
    \end{cases} & X =
    \begin{bmatrix}
        1-t \\
        t \\
        0
    \end{bmatrix} =
    \begin{bmatrix}
        1 \\
        0 \\
        0
    \end{bmatrix} +
    \begin{bmatrix}
        -t \\
        t \\
        0
    \end{bmatrix} =
    \begin{bmatrix}
        1 \\
        0 \\
        0
    \end{bmatrix} + X_0
\end{align*}
Questo ci permette di enunciare il \hl{teorema di costruzione delle soluzioni}.

\paragraph{Nucleo di una matrice} In nucleo di $A = Ker(A)$ è \hl{l'insieme delle
soluzioni del sistema lineare omogeneo associato}
($Ker(A) = \{x \in Mat(m,n;K) \mid AX = 0\} \neq \emptyset$)

\subsection{Teorema di struttura delle soluzioni}
Sia $[A|B]$ risolvibile, \hl{la soluzione del sistema sarà la soluzione particolare di
$[A|B]$ sommata alla soluzione generale del sistema omogeneo associato $[A|0_{m1}]$}

\subsection{Forma chiusa per il calcolo della soluzione di un sistema lineare}
La forma chiusa per \hl{la risoluzione di un generico sistema lineare $AX = B$ è}
\[
    A^{-1}AX = BA^{-1}
\]
Questa forma chiusa è il \hl{teorema di Cramer}

\paragraph{Teorema di Cramer} Se una matrice $A$ è invertibile, allora il
sistema lineare $[A|B]$ associato avrà soluzione $X = A^{-1}B$.

\subparagraph{Dimostrazione} Se $A$ è invertibile, allora $r(A) = n$. Per il
teorema di Rouché-Capelli, la soluzione del sistema associato ad $A$ sarà unica.
Possiamo allora scrivere:
\[
    AX = A(A^{-1}B) = (AA^{-1})B = I_n B = B
\]
Confermando il fatto che $X = A^{-1}B$ è soluzione del sistema.

\subsection{Equivalenza dei sistemi lineari}
Sia $[A|B]$ un generico sistema lineare e $[S|B]$ una sua riduzione a scala.
\hl{I due sistemi avranno le stesse soluzioni}.

\paragraph{Dimostrazione} Per dimostrare il teorema verifichiamo che ogni
operazione di Gauss non modifichi le soluzioni:
\begin{itemize}
    \item Permutazione: modifica solo l'ordine delle equazioni, ma non le soluzioni
    \item Moltiplicazione per scalare: se lo scalare $t \neq 0$ le soluzioni non
        cambiano in quanto
        \[\not{t}(a_{i1}x_1 + \cdots + a_{ij}) = \not{t}(b_i)\]
    \item Somma tra righe: Prendiamo due sistemi così definiti:
        \[
            \begin{cases}
                a_{i1}x_1 + \cdots + a_{in} - b_i = 0 \\
                a_{j1}x_1 + \cdots + a_{jn} - b_j = 0
            \end{cases}
            \begin{cases}
                (a_{i1}x_1 + \cdots + a_{in} - b_i) + t(a_{j1}x_1 + \cdots + a_{jn} - b_j) = 0 \\
                a_{j1}x_1 + \cdots + a_{jn} - b_j = 0
            \end{cases}
        \]
        Siano $(x_1, \ldots, x_n)$ le soluzioni del primo sistema. Allora
        \begin{align*}
            &a_{i1}x_1 + \cdots + a_{in} - b_i = 0 \text{ per ipotesi}\\
            &a_{j1}x_1 + \cdots + a_{jn} - b_j = 0 \text{ per ipotesi}\\
            &(a_{i1}x_1 + \cdots + a_{in} - b_i) + t(a_{j1}x_1 + \cdots + a_{jn} - b_j) = 0
        \end{align*}
        Siano $(x_1, \ldots, x_n)$ le soluzioni anche per il secondo sistema.
        Allora:
        \begin{align*}
            &a_{i1}x_1 + \cdots + a_{in} - b_i = 0 \text{ per soluzione del primo sistema}\\
            &a_{j1}x_1 + \cdots + a_{jn} - b_j = 0 \text{ per ipotesi}\\
            &(a_{i1}x_1 + \cdots + a_{in} - b_i) + t(a_{j1}x_1 + \cdots + a_{jn} - b_j) = 0
        \end{align*}
        Le stesse $(x_1, \ldots, x_n)$ risolvono entrambi i sistemi
\end{itemize}
Le operazioni gaussiane, quindi, non modificano le soluzioni.

\subsection{Algoritmo di Gauss per la risoluzione dei sistemi lineari}
L'algoritmo di Gauss per risolvere i sistemi lineari \hl{si basa sull'equivalenza
dei sistemi lineari}. Consiste nella \hl{riduzione a scala della matrice completa
del sistema lineare}.

\subsection{Teorema di Rouché-Capelli}
Il teorema di Rouché-Capelli ci \hl{permette di capire la risolvibilità di un sistema
lineare in base al rango della sua matrice associata}
($r(A) \leq r([A|B]) \leq r([A]) + 1$). Sia un sistema che ha come matrice dei
coefficienti $[A]$ e matrice completa $[A|B]$. Se:
\begin{itemize}
    \item \hl{$r([A|B]) > r(A)$: il sistema sarà impossibile} Il sistema si dice
        \hl{sovradeterminato}
    \item \hl{$r([A|B]) = r(A) = n$: il sistema avrà un'unica soluzione}
    \item \hl{$r([A|B]) = r(A) < n$: il sistema avrà infinite soluzioni}
\end{itemize}

Consideriamo il sistema associato ad $[A|B] \in Mat(m,n;K)$, per il teorema
sopra enunciato:
\begin{itemize}
    \item La soluzione non esiste se $r([A|B]) > r(A)$
    \item La soluzione esiste unica se $r([A|B]) = r(A) = n$ ($\infty^{0} = 1$
        soluzioni)
    \item Esistono infinite soluzioni dipendenti da $n-r(A)$ parametri se e
        solo se $r([A|B]) = r(A) < n$ ($\infty^{n-r}$ soluzioni)
\end{itemize}
