\documentclass[a4paper,12pt,oneside]{article}
\usepackage[a4paper,margin=1.5cm]{geometry}
\usepackage[italian]{babel}
\usepackage[utf8]{inputenc}
\usepackage[normalem]{ulem}

% Uncomment based on usage
\usepackage{mathtools}
\usepackage{amsmath}
\usepackage{amsfonts}
\usepackage{amssymb}
\usepackage{booktabs}
\usepackage{float}
\usepackage{soulutf8}
%\usepackage{enumerate}
%\usepackage{graphicx}
%\usepackage{wrapfig}
%\usepackage{xcolor}

\title{Appunti Geometria e algebra lineare}
\author{Alexandru Gabriel Bradatan}
\date{Data di compilazione: \today}

\begin{document}

\maketitle

\section{Insiemi}
\hl{Un insieme è una collezione di oggetti}. Tutta la matematica si basa 
sulla teoria assiomatica degli insiemi. \hl{Un insieme A si può indicare
per elencazione ($A = \{a_1, \dots, a_n\}$) o con una condizione 
($A = \{x|\textit{condizione}\}$)}. La \hl{cardinalità di $A$ è il numero di 
oggetti: $|A| = n$}. La cardinalità dell'insieme vuoto è 0.

\paragraph{Esempi} $\mathbb{N} = \{0, 1, 2,\dots\}$, 
$\mathbb{Q} = \{q = \frac{m}{n} | m,n \in \mathbb{Z}, n \neq 0\}$, 
$\mathbb{R} = \{\text{x numeri decimali}\}$.

Un \hl{insieme particolare è l'insieme con nessun elemento detto vuoto, indicato 
con $\emptyset$}. Un altro insieme particolare è \hl{l'insieme di tutti gli tutto
detto insieme universo $U$}.

\subsection{Sottoinsiemi}
Un insieme può essere sottoinsieme di un altro, ossia contenere una parte degli
elementi dell'insieme più grande. Formalizzando si può dire che:
\[
    A \subset B \implies \forall a \in A, a \in B
\]

\subsection{Insiemi numerici}
Trattati nel dettaglio negli appunti di \hl{Analisi 1}.

\subsection{Operazioni}
Le operazioni più usate sono:
\begin{description}
    \item[\hl{Unione}] $A \cup B = \{x | x \in A \vee x \in B\}$
    \item[\hl{Interesezione}] $A \cap B = \{x | x \in A \wedge x \in B\}$
    \item[\hl{Differenza}] $A - B = \{x | x \in A \wedge x \notin B\}$
        Si piò anche trovare indicata con $ \setminus $
    \item[\hl{Prodotto cartesiano}] $A \times B = \{(a,b) | a \in A, b \in B \}$
        Le coppie $(a,b)$ sono anche dette \hl{tuple} (m-uple per $m$ elementi)
\end{description}


\section{Relazioni}
\hl{Una relazione è un sottoinsieme del prodotto cartesiano tra due insiemi}.

Per indicare che due elementi $(a_i, b_j)$ sono legati da una relazione
$R$ usiamo \hl{$a_i \sim_R b_j$}. Per rappresentare le relazioni si possono usare i 
diagrammi di Venn (le patate) con le frecce che collegano i vari elementi tra di
loro.

\paragraph{Esempio} Presi $A = \{a_1, a_2\}, B = \{b_1, b_2\}$, calcoliamo il 
loro prodotto cartesiano e otterremo 16 possibili sottoinsiemi:
\begin{align*}
    R_0 &= \emptyset \\
    R_1 &= \{(a_1, b_1)\}, \dots, R_4 \\
    R_5 &= \{(a_1, b_1), (a_1, b_2)\}, \dots, R_10 \\
    R_{11} &= \{(a_1, b_1), (a_1, b_2), (a_2, b_1)\}, \dots, R_14 \\
    R_{15} &= A \times B
\end{align*}

% tipi di relazioni


\section{Funzioni}
\hl{Le funzioni sono speciali relazioni che associano a ogni elemento del 
primo insieme un elemento del secondo}. Prendiamo
$R_7 = \{(a_1, b_1), (a_2, b_2)\}$. Questa relazione \hl{associa un elemento
del primo insieme a un elemento del secondo} ed è una funzione $f: A \to B$.

\hl{L'insieme A è detto dominio, B il codominio. Se $a \in A$, allora
$b = f(a)$ sarà la sua immagine. L'insieme di tutte le immagini
è detto insieme immagine} e si indica con $Im(f)$. La \hl{controimmagine di b è
quell'elemento tale che $f^{-1}(b) = \{ a \in A \: | \: f(a) = b \}$} (la funzione
inversa in questo caso è solo notazione).

Se \hl{$Im(f) = \text{codominio}$ allora la funzione è suriettiva}.
Se \hl{ad ogni immagine corrisponde una sola controimmagine ($|Im(f^{-1}(b))| = 1$)
allora la funzione è iniettiva}. \hl{Se una funzione è sia iniettiva
che suriettiva è biunivoca}. \hl{Una funzione è invertibile se e solo se
è biunivoca}.

\hl{La funzione $A \times A = \Delta A = Id(A) = \{(a,a) | a \in A\}$ è detta 
funzione identità o insieme diagonale}.

\section{Operazioni}
Le operazioni sono delle speciali funzioni: dati $n+1$ insiemi 
$A_1, \dots, A_{n+1}$ non vuoti, una operazione n-aria $\ast$ è una funzione
che:
\begin{align*}
    A_1 \times \dots \times A_n &\to A_{n+1} \\
    (a_1, \dots, a_n) &\mapsto \ast (a_1, \dots, a_n)
\end{align*}

Se \hl{gli insiemi usati nel prodotto cartesiano sono lo stesso insieme $A$
si dice che l'operazione è interna} (esempio: la somma). Se il \hl{numero di insiemi è 2 si dice binaria}.

Esempi di operazioni:
\begin{description}
    \item[Somma] \[
        \begin{array}{ccc}
            \mathbb{N} \times \mathbb{N} &\to &\mathbb{N} \\
            (n1, n2) &\mapsto &n3 = n1 + n2
        \end{array} 
    \]
    \item[Differenza] \[
        \begin{array}{ccc}
            \mathbb{N} \times \mathbb{N} &\to &\mathbb{Z} \\
            (n1, n2) &\mapsto &n3 = n1 - n2 
        \end{array}
    \]
\end{description}

Le varie operazioni possono essere rappresentate in tabelle che indicano tutti i
posibili casi. Ad esempio, esistono $2^4 = 16$ diverse operazioni binarie 
interne ad $A = \{a_1, a_2\}$.

\begin{table}[H]
    \centering
    \begin{tabular}{c|cc}
        $\ast$ & $a_1$ & $a_2$ \\
        \midrule
        $a_1$ & $a_1$ & $a_2$ \\
        $a_2$ & $a_1$ & $a_2$ \\
    \end{tabular}
    \caption{Esempio di operazione interna binaria ad A}
\end{table}

\end{document}
