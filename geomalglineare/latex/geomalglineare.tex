\documentclass[a4paper,12pt,oneside]{article}
\usepackage[a4paper,margin=1.5cm]{geometry}
\usepackage[italian]{babel}
\usepackage[utf8]{inputenc}
\usepackage[normalem]{ulem}

% Uncomment based on usage
\usepackage{mathtools}
\usepackage{amsmath}
\usepackage{amsfonts}
\usepackage{amssymb}
\usepackage{booktabs}
\usepackage{float}
%\usepackage{enumerate}
%\usepackage{graphicx}
%\usepackage{wrapfig}
%\usepackage{xcolor}

\title{Appunti Geometria e algebra lineare}
\author{Alexandru Gabriel Bradatan}
\date{}

\begin{document}

\maketitle

\section{Insiemi}
\textbf{Un insieme è una collezione di oggetti}. Tutta la matematica si basa 
sulla teoria assiomatica degli insiemi.

\textbf{Un insieme A si indica}: \( A = {a1,..., an} \). La \textbf{cardinalità di 
A è il numero di oggetti}: \(|A| = n\). La cardinalità dell'insieme vuoto è 0.

Esempi di insiemi:
\begin{itemize}
    \item \( \mathbb{N} = \{0, 1, 2,\dots\} \)
    \item \( \mathbb{Z} = \{\dots, -1, 0, 1,\dots\} \)
    \item \( \mathbb{Q} = \{q = \frac{m}{n} \: | \: m,n \in \mathbb{Z}, n \neq 0\} \) 
        (costruzione con condizione)
    \item \( \mathbb{R} = \{\text{x numeri decimali}\} \)
    \item \( \mathbb{C} = \{z = x + iy \: | \: x,y \in \mathbb{R}, i^2 = -1\} \)
\end{itemize}

\subsection{Operazioni tra insiemi}
Ci sono \textbf{unione} (\(\cup\)), \textbf{intersezione} (\(\cap\)) e 
\textbf{differenza} (\(\backslash\)) e \textbf{prodotto cartesiano} (\(\times\)).

\begin{description}
    \item[Unione] Prendo tutti gli elementi in A e B
    \item[Interesezione] Prendo tutti gli elementi che sono sia in A che in B
    \item[Differenza] Prendo tutti gli elementi che sono in A ma non in B
    \item[Prodotto cartesiano] Insieme di \textbf{m-uple} (m-uple: 
        \((a_1, a_2, \dots, a_n)\)) contenenenti tutte le combinazioni degli 
        elementi di A e B
\end{description}

\section{Relazioni}
\textbf{Una relazione è un sottoinsieme del prodotto cartesiano tra due insiemi.}

Esempio:
\begin{align*}
    A &= \{a_1, a_2\}, \: B = \{b_1, b_2\} \\
    A &\times B = \{\text{tutte le combinazioni degli elementi di A e B}\} \\
    R_0 &= \emptyset \\
    R_1 &= \{(a_1, b_1)\}, \dots, R_4 \\
    R_5 &= \{(a_1, b_1), (a_1, b_2)\}, \dots, R_10 \\
    R_{11} &= \{(a_1, b_1), (a_1, b_2), (a_2, b_1)\}, \dots, R_14 \\
    R_{15} &= \{A \times B\}
\end{align*}

Dati due elementi \((a_i, b_j)\) appartenenti tutti e due ad una stessa relazione
\(R\), si può scrivere che \(a_i \sim_R b_j\). Per rappresentare le relazioni
si possono usare i diagrammi di Venn (le patate).

\section{Funzioni}
\textbf{Le funzioni sono speciali relazioni che associano a ogni elemento del 
primo insieme un elemento del secondo}. Prendiamo
\(R_7 = \{(a_1, b_1), (a_2, b_2)\}\). Questa relazione \textbf{associa un elemento
del primo insieme a un elemento del secondo} ed è una funzione \(f: A \to B\).

\textbf{L'insieme A è detto dominio, B il codominio}. Se \(a \in A\), allora
\(b = f(a)\) \textbf{sarà la sua immagine}. \textbf{L'insieme di tutte le immagini
è detto insieme immagine} e si indica con \(Im(f)\). La \textbf{controimmagine di b è
quell'elemento tale che \(f^{-1}(b) = \{ a \in A \: | \: f(a) = b \}\)} (la funzione
inversa in questo caso è solo notazione).

Se \textbf{\(Im(f) = \text{codominio}\) allora la funzione è suriettiva}.
Se \textbf{ad ogni immagine corrisponde una sola controimmagine (\(|Im(f^{-1}(b))| = 1\))
allora la funzione è iniettiva}. \textbf{Se una funzione è sia iniettiva
che suriettiva è biunivoca}. \textbf{Una funzione è invertibile se e solo se
è biunivoca}.

\textbf{La funzione \(A \times A = \Delta A = Id(A) = \{(a,a) | a \in A\}\) è detta 
funzione identità o insieme diagonale}.

\section{Operazioni}
Le operazioni sono delle speciali funzioni: dati \(n+1\) insiemi 
\(A_1, \dots, A_{n+1}\) non vuoti, una operazione n-aria \(\ast\) è una funzione
che:
\begin{align*}
    A_1 \times \dots \times A_n &\to A_{n+1} \\
    (a_1, \dots, a_n) &\mapsto \ast (a_1, \dots, a_n)
\end{align*}

Se \textbf{gli insiemi usati nel prodotto cartesiano sono lo stesso insieme \(A\)
si dice che l'operazione è interna} (esempio: la somma). Se il \textbf{numero di insiemi è 2 si dice binaria}.

Esempi di operazioni:
\begin{description}
    \item[Somma] \(\mathbb{N} \times \mathbb{N} \to \mathbb{N}, (n1, n2) \mapsto n3 = n1 + n2 \)
    \item[Differenza] \(\mathbb{N} \times \mathbb{N} \to \mathbb{Z}, (n1, n2) \mapsto n3 = n1 - n2 \)
\end{description}

Le varie operazioni possono essere rappresentate in tabelle che indicano tutti i
posibili casi. Ad esempio, esistono \(2^4 = 16\) diverse operazioni binarie 
interne ad \(A = \{a_1, a_2\}\).

\begin{table}[H]
    \centering
    \begin{tabular}{c|cc}
        \(\ast\) & \(a_1\) & \(a_2\) \\
        \midrule
        \(a_1\) & \(a_1\) & \(a_2\) \\
        \(a_2\) & \(a_1\) & \(a_2\) \\
    \end{tabular}
    \caption{Esempio di operazione interna binaria ad A}
\end{table}

\end{document}
