\documentclass[a4paper,12pt,oneside]{article}
\usepackage[a4paper,margin=1.5cm]{geometry}
\usepackage[italian]{babel}
\usepackage[utf8]{inputenc}
\usepackage[normalem]{ulem}

% Uncomment based on usage
\usepackage{mathtools}
\usepackage{amsmath}
\usepackage{amsfonts}
\usepackage{amssymb}
\usepackage{booktabs}
\usepackage{float}
%\usepackage{enumerate}
%\usepackage{graphicx}
%\usepackage{wrapfig}
%\usepackage{xcolor}

\title{Appunti Analisi 1}
\author{Alexandru Gabriel Bradatan}
\date{}

\begin{document}

\maketitle

\section{Insiemi}
Non viene definito (\textbf{concetto primitivo}): una collezione, famiglia,
classe di oggetti (non necessariamente numeri). Indicato solitamente con una
lettera maiuscola.

\textbf{Rappresentati per}:
\begin{itemize}
    \item elencazione: \(A = \{a, b, c\}\)
    \item condizione: \(A = \{lettere alfabeto\}\)
\end{itemize}

Un oggetto può appartenere ($\in$) o non appartenere ($\notin$) ad un insieme.

\subsection{Uguaglianza}
A e B sono uguali ($A = B$) se hanno gli stessi elementi.
\textit{Osserva: $A = B \iff \subseteq B \wedge B \subseteq A$}

\subsection{Inclusione}
A può essere contenuto o uguale in B ($A \subseteq B or A \subset B$).
A è un sotto insieme di B se $\forall a \in A a \in B$. Il simbolo è detto
\textbf{inclusione. L'inclusione è una relazione d'ordine}.

\subsection{Operazioni sugli insiemi}
Le operazioni sono unione, intersezione, differenza, prodotto cartesiano,
insieme complementare.

\begin{description}
    \item[Unione]: $A \cup B = \{ x \in U | x \in A \vee x \in B \}$
    \item[Intersezione]: $A \cap B = \{ x \in U | x \in A \wedge x \in B \}$
    \item[Complementare]: $A^c = \bar{A} = \{ x \in U | x \notin A \}$
    \item[Differenza]: $A \setminus B = \{ x \in U | x \in A \wedge x \notin B \}$
    \item[Prodotto cartesiano]: $A \times B = \{ (x,y) | x \in A \wedge x \in B \}$
\end{description}

\textbf{Intersezione e unione sono commutative e associative.}

\subsection{Insiemi particolari}
\begin{itemize}
    \item \textbf{Insieme vuoto}: $\emptyset$
    \item \textbf{Insieme universo}: $U$, contiene tutto
\end{itemize}

\section{Numeri}
\subsection{Numeri naturali}
Sono tutti i numeri \textbf{interi positivi incluso lo 0}.
\[
    \mathbb{N} = \{0, 1, 2, 3, \dots \}
\]
Può \textbf{essere costruito a partire da un solo numero}: basta aggiungere un'unità
ogni volta.

Ha la \textbf{proprietà di contenere sempre il successore a un numero}: ci
permette di usare il \textbf{principio di induzione}. \textbf{Tutti i
sottoinsiemi di $\mathbb{N}$ godono del principio del minimo intero. Poiché è
valido il principio del minimo intero, $\mathbb{N}$ è un insieme ben ordinato}.

\subsubsection{Il principio di induzione}
Sia $S \subseteq \mathbb{N}$ un sottoinsieme tale che:
\begin{itemize}
    \item $0 \in S$
    \item $\forall n \in S \implies n+1 \in S$ (S ha sempre un successore)
\end{itemize}

Allora S coincide con N.

Il \textbf{principio di induzione ha una traduzione in termini logici}. Il
principio di induzione può essere usato per dimostrare teoremi in N.


\end{document}
