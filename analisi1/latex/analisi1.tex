\documentclass[a4paper,12pt,oneside]{article}
\usepackage[a4paper,margin=1.5cm]{geometry}
\usepackage[italian]{babel}
\usepackage[utf8]{inputenc}
\usepackage[normalem]{ulem}

% Uncomment based on usage
\usepackage{mathtools}
\usepackage{amsmath}
\usepackage{amsfonts}
\usepackage{amssymb}
\usepackage{booktabs}
\usepackage{float}
\usepackage{soulutf8}
%\usepackage{enumerate}
%\usepackage{graphicx}
%\usepackage{wrapfig}
%\usepackage{xcolor}

\title{Appunti Analisi 1}
\author{Alexandru Gabriel Bradatan}
\date{Data compilazione: \today}

\begin{document}

\maketitle

\section{Insiemi}
Non viene definito (\hl{concetto primitivo}): una collezione, famiglia,
classe di oggetti (non necessariamente numeri). Indicato solitamente con una
lettera maiuscola.

\hl{Rappresentati per}:
\begin{itemize}
    \item elencazione: \(A = \{a, b, c\}\)
    \item condizione: \(A = \{lettere alfabeto\}\)
\end{itemize}

Un oggetto può appartenere ($\in$) o non appartenere ($\notin$) ad un insieme.

\subsection{Uguaglianza}
A e B sono uguali ($A = B$) se hanno gli stessi elementi.
\textit{Osserva: $A = B \iff \subseteq B \wedge B \subseteq A$}

\subsection{Inclusione}
A può essere contenuto o uguale in B ($A \subseteq B or A \subset B$).
A è un sotto insieme di B se $\forall a \in A a \in B$. Il simbolo è detto
\hl{inclusione. L'inclusione è una relazione d'ordine}.

\subsection{Operazioni sugli insiemi}
Le operazioni sono unione, intersezione, differenza, prodotto cartesiano,
insieme complementare.

\begin{description}
    \item[Unione]: $A \cup B = \{ x \in U | x \in A \vee x \in B \}$
    \item[Intersezione]: $A \cap B = \{ x \in U | x \in A \wedge x \in B \}$
    \item[Complementare]: $A^c = \bar{A} = \{ x \in U | x \notin A \}$
    \item[Differenza]: $A \setminus B = \{ x \in U | x \in A \wedge x \notin B \}$
    \item[Prodotto cartesiano]: $A \times B = \{ (x,y) | x \in A \wedge x \in B \}$
\end{description}

\hl{Intersezione e unione sono commutative e associative.}

\subsection{Insiemi particolari}
\begin{itemize}
    \item \hl{Insieme vuoto}: $\emptyset$
    \item \hl{Insieme universo}: $U$, contiene tutto
\end{itemize}

\section{Numeri}
\subsection{Numeri naturali}
Sono tutti i numeri \hl{interi positivi incluso lo 0}.
\[
    \mathbb{N} = \{0, 1, 2, 3, \dots \}
\]
Può \hl{essere costruito a partire da un solo numero}: basta aggiungere un'unità
ogni volta.

Ha la \hl{proprietà di contenere sempre il successore a un numero}: ci
permette di usare il \hl{principio di induzione}. \hl{Tutti i sottoinsiemi di 
$\mathbb{N}$ godono del principio del minimo intero. Poiché è valido il 
principio del minimo intero, $\mathbb{N}$ è un insieme ben ordinato}.

\paragraph{Il principio di induzione} Sia \hl{$S \subseteq \mathbb{N}$ un 
sottoinsieme tale che}:
\begin{itemize}
    \item \hl{$0 \in S$}
    \item \hl{$\forall n \in S \implies n+1 \in S$} (S ha sempre un successore)
\end{itemize}

Allora \hl{S coincide con $\mathbb{N}$}.

Il \hl{principio di induzione ha una traduzione in termini logici}. Il
principio di induzione può essere usato per dimostrare teoremi in $\mathbb{N}$.

\paragraph{Il principio di induzione (logica)} Sia $P(n)$ un predicato 
(proposizione) che dipende da $n \in \mathbb{N}$ tale che:
\begin{itemize}
    \item quando \hl{$P(0)$ è vero}
    \item \hl{$\forall n \in \mathbb{N} P(n) \implies P(n+1)$: assumendo $P(n)$ come
        vero, riesco a dimostrare che il successore è vero}
\end{itemize}

\paragraph{Esempio} Dimostra $P(n) = 2^n > n \forall n \in \mathbb{N}$. $P(0)$ è vera: 
$2^0 > 0$. Suppongo che $P(n)$ sia vera, dimostro $P(n+1)$: $2^n \cdot 2 > 2n \geq n+1$
quindi $P(n+1)$ è vera. Se $P(n+1)$ è vera, allora vale il principio di induzione
e tutto il predicato è vero in $\mathbb{N}$.

\paragraph{Principio del minimo intero} \hl{Ogni sottoinsieme di 
$\mathbb{N}$ ha un elemento minimo (più piccolo di tutti gli altri)}.

\paragraph{Definizioni delle operazioni in $\mathbb{N}$}
\begin{description}
    \item[Somma] 
        \begin{align*}
            + &: \mathbb{N} \times \mathbb{N} \to \mathbb{N} \\
            somma(n1, n2) &\to n_1 + n_2 \in \mathbb{N}
        \end{align*}
    \item[Prodotto] 
        \begin{align*}
            * &: \mathbb{N} \times \mathbb{N} \to \mathbb{N} \\
            prodotto(n1, n2) &\to n1 \cdot n2 \in \mathbb{N}
        \end{align*}
\end{description}

\paragraph{Proprietà delle operazioni}
\begin{description}
    \item[commutativa] $n_1 + n_2 = n_2 + n_1$
    \item[associativa] $n_1 + (n_2 + n_3) = (n_1 + n_2) + n_3$
    \item[distributiva] $n_1 \cdot (n_2 + n_3) = n_1 \cdot n_2 + n_1 \cdot n_3$
\end{description}

\subsubsection{Sommatoria}
Si indica con la \hl{sigma maiuscola}:
\[
    \sum_{i \in I} a_i
\]
Dove:
\begin{itemize}
    \item $I$ è un \hl{insieme finito}. I suoi elementi sono chiamati 
        \hl{indici} (segnaposti: indicano una posizione)
    \item $(a_i)_{a \in I}$ è una \hl{famiglia di numeri che dipendono da $i$}
\end{itemize}

\paragraph{Esempio} Dati $I = {1, 2, 3}, a_i = 2^i$, possiamo scrivere 
$\sum_{i \in I} a_i = 2^1 + 2^2 + 2^3$

\paragraph{Alcune sommatorie famose}
\begin{description}
    \item[Formula di Gauss] $\sum_{i=1}^n (i) = \frac{n\cdot(n-1)}{2}$
    \item[Somma di una progressione geometrica]
        \begin{align*}
            \sum_{i = 0}^n a_i &= \sum_{i = 0}^n aq^i = a \sum_{i = 0}^n q^i = \\
            &= a(\frac{1-q^n+1}{1-q}) \\
            \text{se } q = 1 \: &= a(n+1)
        \end{align*}
        
        Dimostrazione:
        \begin{align*}
            \not{a} \sum_{i=0}^n q^i &= \not{a}\left(\frac{1-q^n+1}{1-q}\right) \\
            \text{Dimostriamo che: } (1-q)\sum_{i=0}^n q^i &= 1-q^n+1 \\
            (1-q)\sum_{i=0}^n q^i &= \sum_{i=0}^n q^i - q\sum_{i=0}^n q^i \\
            &= \sum_{i=0}^n q^i - \sum_{i = 0}^{n} q^{i+1} \\
            &= \sum_{i=0}^n q^i - \sum_{i=1}^{n+1} q^i \\
            &= (q^0 + \sum_{i=1}^{n+1} q^i) - (\sum_{i=0}^n q^i + q^{n+1}) \\
            &= 1 + \sum_{i=1}^{n+1} q^i) - \sum_{i=0}^n q^i - q^{n+1} \\
            &= 1 - q^{n+1} \implies \text{ è dimostrato}
        \end{align*}
\end{description}

\paragraph{Le proprietà della sommatoria} 
\begin{itemize}
    \item La sommatoria è \hl{un operatore lineare}
    \item \hl{l'indice è muto}: non importa il nome dell'indice 
    \item \hl{traslando gli indici, la sommatoria non cambia}: è importante 
        che il numero di elementi sia uguale
    \item \hl{si definiscono sommatorie anche su due o più famiglie di 
        indici}: prima sommo una famiglia, poi l'altra: $\sum_{i \in I, j \in J} a_{ij}$
        
        Esempio: $\sum_{i \in I, j \in J} (i)^J = \sum_{i=1}^{2} (\sum_{j=0}^{3} (i)^j)$
    \item \hl{vale la proprietà dissociativa}:
        $\sum_{i \in I} (a_i + b_i) = \sum_{i \in I} (a_i) + \sum_{i \in I} (b_i)$
    \item \hl{le costanti possono essere portate fuori}:
        $\sum_{i \in I} Ka_i = K\cdot\sum_{i \in I} a_i$
    \item \hl{scomposizione di una sommatoria}:
        $\sum_{i=1}^n a_i = \sum_{i=1}^k a_i + \sum_{i=k+1}^n a_i$
    \item \hl{riflessione degli indici}:
        $\sum_{i=0}^n = \sum_{i=0}^n a_{n-i}$
\end{itemize}

\subsubsection{La produttoria}
Si indica con un grande pi greco. E' uguale alla sommatoria ma al posto di
fare la somma fa il prodotto.

\paragraph{Proprietà}
\begin{itemize}
    \item $\prod_{i \in I} ka_i = k^{\#i}\prod_{i \in I} a_i$
    - Non vale la dissociativa
\end{itemize}
\end{document}
