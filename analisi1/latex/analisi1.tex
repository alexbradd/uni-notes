\documentclass[a4paper,11pt,oneside,dvipsnames]{article}

\usepackage[a4paper,margin=1.5cm]{geometry}
\usepackage[italian]{babel}
\usepackage[utf8]{inputenc}
\usepackage[T1]{fontenc}
\usepackage{lmodern}

\usepackage{mathtools}
\usepackage{amsmath}
\usepackage{amsfonts}
\usepackage{amssymb}
\usepackage{xfrac}

\usepackage{booktabs}
\usepackage{float}
\usepackage{soulutf8}
\usepackage[usenames,dvipsname]{xcolor}
% \usepackage[normalem]{ulem}

%\usepackage{enumerate}
%\usepackage{graphicx}
%\usepackage{wrapfig}

\sethlcolor{Goldenrod}

\title{Appunti di analisi 1}
\author{Alexandru Gabriel Bradatan}
\date{Data compilazione: \today}

\begin{document}

\maketitle
\tableofcontents

\section{Insiemi}
Vedi \hl{appunti di geometria e algebra lineare}.

\section{Insiemi numerici}
\subsection{Numeri naturali ($\mathbb{N}$)}
Sono i numeri \hl{interi positivi incluso lo 0}. Può \hl{essere costruito a 
partire da un solo numero (lo 0)} basta aggiungendo un'unità ogni volta.
\[ \mathbb{N} = \{0, 1, 2, 3, \dots \} \]

\subsubsection{Proprietà} 
\hl{Contiene sempre il successore di ogni suo elemento 
(principio di induzione)}. Gode della \hl{relazione d'ordine $\leq$, il che lo 
rende un insieme ordinato}. $\mathbb{N}$, come tutti i suoi sottoinsiemi, 
\hl{godono del principio del minimo intero} che lo rende, insieme ai suoi 
sottoinsiemi, \hl{un insieme \textit{ben} ordinato}.

\subsubsection{Operazioni definite} 
In $\mathbb{N}$ sono definite somma e prodotto: in questo modo:\[
    \begin{array}{cccc}
        +: &\mathbb{N} \times \mathbb{N} &\to &\mathbb{N} \\
        &somma(n1, n2) &\mapsto & n_1 + n_2 \in \mathbb{N}
    \end{array} \quad
    \begin{array}{cccc}
        \cdot: &\mathbb{N} \times \mathbb{N} &\to &\mathbb{N} \\
        &prodotto(n1, n2) &\mapsto & n_1 \cdot n_2 \in \mathbb{N}
    \end{array}
\]

\paragraph{Proprietà delle operazioni}
\begin{description}
    \item[Commutativa] $n_1 + n_2 = n_2 + n_1$
    \item[Associativa] $n_1 + (n_2 + n_3) = (n_1 + n_2) + n_3$
    \item[Distributiva] $n_1 \cdot (n_2 + n_3) = n_1 \cdot n_2 + n_1 \cdot n_3$
\end{description}

\subsubsection{Principio del minimo intero} 
\hl{Ogni sottoinsieme di $\mathbb{N}$ ha un elemento minimo (più piccolo di 
tutti gli altri)}.

\subsubsection{Il principio di induzione} 
Sia \hl{$S \subseteq \mathbb{N}$ un sottoinsieme tale che 
$0 \in S$ e $\forall n \in S \implies n+1 \in S$}. Allora 
\hl{S coincide con $\mathbb{N}$}.

\paragraph{Il principio di induzione nella logica} Il principio di induzione 
può essere usato per dimostrare teoremi in $\mathbb{N}$. Enunciamolo in questo 
modo: sia \hl{$P(n)$ un predicato che dipende da $n \in \mathbb{N}$} tale che
\hl{$P(n_0)$ sia vero e che $\forall n \in \mathbb{N} P(n) \implies P(n+1)$}. Il
predicato \hl{sarà vero per tutti gli $n \geq n_0$}.

\subsubsection{Fattoriale} 
Preso $n \in N$, il fattoriale di $n$ sarà
$n! = 1 \cdot 2 \cdot 3 \cdot \ldots \cdot (n-1) \cdot n$. Una eccezione è lo 0:
il fattoriale di 0 è $0! = 1$. Il fattoriale è un numero definito che può essere
definito induttivamente: $n! = n(n-1)!$.

\subsubsection{Coefficiente binomiale} 
Già incontrati nella probabilità:
\[ \binom{n}{n} = \frac{n!}{k!(n-k)!} \]
con $n \in \mathbb{N}, 0 \leq k \leq n$. Convenzionalmente $\binom{0}{0} = 1$. 
Il coefficiente binomiale viene usato nel binomio di Newton.

\subsubsection{Binomio di Newton} 
Il binomio di Newton ci permette di calcolare l'elevamento a qualsiasi potenza 
di un binomio:
\[ (a+b)^n = \sum_{k=0}^n \binom{n}{k} a^{n-k} b^k \]
La formula è dimostrabile per induzione (se sei proprio interessato, vedi gli
appunti a penna).

\subsection{Numeri interi relativi ($\mathbb{Z}$)}
\'E l'insieme $\mathbb{Z}$. \hl{Non esiste un minimo, di conseguenza non valgono 
il principio del minimo intero e il principio di induzione}. \'E \hl{definita la
relazione d'ordine $\leq$, quindi è un insieme ordinato} ma a causa della
mancata validità dei due principi nominati precedentemente, \hl{non è un insieme
ben ordinato}. $\mathbb{Z}$ è, inoltre, più grande di $\mathbb{N}$: $ N \subset Z$.

\subsubsection{Costruzione} 
Per costruire il numeri relativi, \hl{definiamo una relazione di equivalenza 
$\sim$ in $\mathbb{N} \times \mathbb{N}$ tale che}:
\[ (a,b) \sim (h,k) \iff a+k=b+h \]
Questa relazione di equivalenza \hl{ci permette di descrivere tutti i numeri negativi
che sono la differenza dei numeri $a$ e $b$ o $h$ e $k$}: per esempio \hl{$-1$ è la 
classe di equivalenza $[(2,3)]_{\sim}$}. $\mathbb{Z}$ viene, quindi, definito come
$\mathbb{Z} = (\mathbb{N} \times \mathbb{N}) / \sim$

\paragraph{Dimostrazione che $\sim$ è una relazione di equivalenza} Per
dimostrare che $\sim$ è una relazione di equivalenza, verifichiamo che soddisfi
i requisiti:
\begin{itemize}
    \item è riflessiva: $(m,n) \sim (m,n) \implies m+n = n+m$
    \item è simmetrica: $(a,b) \sim (c,d) = (c,d) \sim (a,b)$
    \item è transitiva: $(a,b) \sim (c,d), (c,d) \sim (e,f) \implies (a,b)\sim(e,f)$
        Infatti: 
        \begin{align*}
            a+d=b+c, &\quad c+f=d+e \\
            a-b=c-d, &\quad c-d=e-f \\
            a-b&=e-f \\
            a+f&=b+e
        \end{align*}
\end{itemize}

\subsubsection{Operazioni definite} 
Le operazioni sono \hl{le stesse di $\mathbb{N}$ ma aggiornate}:
\[
    \begin{array}{cccc}
        +: &\mathbb{Z} \times \mathbb{Z} &\to &\mathbb{Z} \\
        &((a, b)_{\sim}, (h,k)_{\sim}) &\mapsto & (a+h, b+k)_{\sim}
    \end{array} \quad
    \begin{array}{cccc}
        \cdot: &\mathbb{Z} \times \mathbb{Z} &\to &\mathbb{Z} \\
        &((a, b)_{\sim}, (h,k)_{\sim}) &\mapsto & (ah + bk, bh + ak)_{\sim}
    \end{array}
\]

\paragraph{Proprietà delle operazioni} Mantnengono le stesse proprietà che 
avevano in $\mathbb{N}$.

\subsection{Numeri razionali ($\mathbb{Q}$)}
\'E l'insieme $\mathbb{Q}$. \hl{Non esiste un minimo, di conseguenza non valgono 
il principio del minimo intero e il principio di induzione}. \'E \hl{definita la
relazione d'ordine $\leq$, quindi è un insieme ordinato} ma a causa della
mancata validità dei due principi nominati precedentemente, \hl{non è un insieme
ben ordinato}.

\subsubsection{Costruzione}
Per costruire il numeri razionali, \hl{definiamo una relazione di equivalenza 
$\approx$ in $\mathbb{Z} \times (\mathbb{Z} - \{0\})$ tale che}:
\[ (a,b) \approx (h,k) \iff ak=bh \]
Questa relazione di equivalenza \hl{ci permette di descrivere tutti i numeri razionali
che sono divisione dei numeri $a$ e $b$ o $h$ e $k$}: per esempio \hl{$\sfrac{2}{3}$
è la classe di equivalenza $[(2,3)]_{\approx}$}. $\mathbb{Q}$ viene, quindi, 
definito come $\mathbb{Z} = (\mathbb{Z} \times (\mathbb{Z} - \{0\}) / \approx$

\paragraph{Dimostrazione che $\approx$ è una relazione di equivalenza} Per
dimostrare che $\approx$ è una relazione di equivalenza, verifichiamo che soddisfi
i requisiti:
\begin{itemize}
    \item è riflessiva: $(m,n) \approx (m,n) \implies mn = nm$
    \item è simmetrica: $(a,b) \approx (c,d) = (c,d) \approx (a,b)$
    \item è transitiva: $(a,b) \approx (c,d), (c,d) \approx (e,f) \implies (a,b)\approx(e,f)$
        Infatti: 
        \begin{align*}
            ad=bc, &\quad cf=de \\
            \frac{a}{b}=\frac{c}{d}, &\quad \frac{c}{d}=\frac{e}{f} \\
            \frac{a}{b}&=\frac{e}{f} \\
            af&=be
        \end{align*}
\end{itemize}

\subsubsection{Operazioni definite} 
Le operazioni sono \hl{le stesse che sono definite in $\mathbb{Z}$ ma aggiornate}:
\[
    \begin{array}{cccc}
        +: &\mathbb{Q} \times \mathbb{Q} &\to &\mathbb{Q} \\
        &((a, b)_{\approx}, (h,k)_{\approx}) &\mapsto & (ak+bh, b+k)_{\sim}
    \end{array} \quad
    \begin{array}{cccc}
        \cdot: &\mathbb{Q} \times \mathbb{Q} &\to &\mathbb{Q} \\
        &((a, b)_{\approx}, (h,k)_{\approx}) &\mapsto & (ah, bk)_{\approx}
    \end{array}
\]

\paragraph{Proprietà delle operazioni} Mantengono le stesse proprietà che 
avevano in $\mathbb{Z}$.

\subsubsection{La rappresentazione decimale}
La rappresentazione decimale di un numero non è nient'altro che un \hl{allineamento
di cifre}. Le rappresentazioni decimali che si trovano \hl{nei razionali sono 
limitate, illimitate periodiche}. Esistono anche rappresentazioni \hl{illimitate,
ma non sono contenute in $\mathbb{Q}$}.

\'E costituita da una \hl{parte intera (necessariamente finita)} e una \hl{parte
decimale che può essere finita o illimitata} (si ricorda che in $\mathbb{Q}$ solo
illimitati periodici). Può essere scritta come:
\[ x = \pm \sum_{j=0}^k c_j \cdot 10^j + \sum_{l=0}^m d_l 10^{-l} \]
Dove la prima sommatoria rappresenta la parte intera e la seconda la parte 
decimale.

\subsection{I numeri reali ($\mathbb{R}$)}
L'insieme dei numeri reali contiente \hl{qualsiasi rappresentazione decimale
possibile, limitata o illimitata}. Di conseguenza, $\mathbb{R}$ contiene tutti
gli insiemi visti fino ad ora. Nell'insieme dei reali è \hl{definita la relazione
d'ordine $\leq$, rendolo un insieme ordinato}. Inoltre, vale anche \hl{l'assioma
di completezza, che rende $\mathbb{R}$ un insieme ordinato e completo}.

\subsubsection{Operazioni definite}
Le operazioni definite sono \hl{sempre le stesse trovate negli insiemi precedenti}:
\[
    \begin{array}{cccc}
        +: &\mathbb{R} \times \mathbb{R} &\to &\mathbb{R} \\
        &somma(a,b) &\mapsto & a + b \in \mathbb{R}
    \end{array} \quad
    \begin{array}{cccc}
        \cdot: &\mathbb{R} \times \mathbb{R} &\to &\mathbb{R} \\
        &prodotto(a,b) &\mapsto & a \cdot b \in \mathbb{R}
    \end{array}
\]

\paragraph{Proprietà delle operazioni} Mantengono le stesse proprietà che 
avevano in $\mathbb{Q}$.

\subsubsection{Assioma di completezza}
Siano \hl{$A, B \subseteq R$ tali che}:
\begin{itemize}
    \item $A,B \neq \emptyset$
    \item $A \cap B = \emptyset$ 
    \item $A \cup B = R$
    \item $\forall a \in A, \forall b \in B \; a < b$ 
\end{itemize}
allora esiste un unico numero reale tale che 
\hl{$\forall a \in A, \forall b \in B \; a \leq s \leq b$}. $s$ è detto elemento 
\hl{separatore}. 

\subsection{Numeri complessi}
\'E l'insieme che \hl{completa i numeri reali}: \hl{ci permettono di risolvere le
equazioni polinomiali che non riuscivamo nei reali (chiusura algebrica)}. 

\subsubsection{Costruzione}
\'E un \hl{insieme di coppie ordinate di numeri reali} appartenenti a 
$\mathbb{R} \times \mathbb{R}$.

\subsubsection{Operazioni}
Le operazioni sono \hl{sempre le stesse che in $\mathbb{R}$ ma aggiornate}:
\begin{align*}
    \begin{array}{cccc}
        +: &\mathbb{C} \times \mathbb{C} &\to &\mathbb{C} \\
        &somma((a,b), (c,d)) &\mapsto & (a+c, b+d) \in \mathbb{R}
    \end{array} \\
    \begin{array}{cccc}
        \cdot: &\mathbb{C} \times \mathbb{C} &\to &\mathbb{C} \\
        &prodotto((a,b), (c,d)) &\mapsto & (ac-bd, ad + bc) \in \mathbb{R}
    \end{array}
\end{align*} 

\section{Sommatoria e produttoria}
\subsection{Sommatoria}
Si indica con la \hl{sigma maiuscola}:
\[ \sum_{i \in I}^n a_i \]
Dove:
\begin{itemize}
    \item $I$ è un \hl{insieme finito}. I suoi elementi sono chiamati \hl{indici}
    \item $(a_i), i \in I$ è una \hl{famiglia di numeri che dipendono da $i$}
\end{itemize}

\subsubsection{Le proprietà della sommatoria} 
\begin{itemize}
    \item La sommatoria è \hl{un operatore lineare}
    \item \hl{l'indice è muto}: non importa il nome dell'indice 
    \item \hl{traslando gli indici, la sommatoria non cambia}: è importante 
        che il numero di elementi sia uguale
    \item \hl{si definiscono sommatorie anche su due o più famiglie di indici}: 
        $\sum_{i \in I, j \in J} a_{ij} = \sum_{i \in I}\sum_{j \in J} a_{ij}$
    \item \hl{vale la proprietà dissociativa}:
        $\sum_{i \in I} (a_i + b_i) = \sum_{i \in I} (a_i) + \sum_{i \in I} (b_i)$
    \item \hl{le costanti possono essere portate fuori}:
        $\sum_{i \in I} Ka_i = K\cdot\sum_{i \in I} a_i$
    \item \hl{può essere scomposta in sommatorie più piccole}:
        $\sum_{i=1}^n a_i = \sum_{i=1}^k a_i + \sum_{i=k+1}^n a_i$
    \item \hl{riflessione degli indici}:
        $\sum_{i=0}^n = \sum_{i=0}^n a_{n-i}$
\end{itemize}

\subsubsection{Alcune sommatorie famose}
\begin{description}
    \item[Formula di Gauss] $\sum_{i=1}^n (i) = \frac{n\cdot(n-1)}{2}$
    \item[Somma di una progressione geometrica]
        \begin{align*}
            \sum_{i = 0}^n q^i &= \frac{1-q^{n+1}}{1-q} \\
            &= n+1 \text{ per } q = 1
        \end{align*}
        Dimostrazione:
        \begin{align*}
            \text{Tesi: } (1-q)\sum_{i = 0}^n q^i &= 1-q^{n+1} \\
            (1-q)\sum_{i = 0}^n q^i &= \sum_{i = 0}^n q^i - q \sum_{i = 0}^n q^i \\
            &= \sum_{i = 0}^n q^i - \sum_{i = 0}^n q^{i+1} \text{ prendiamo } k = i + 1 \\
            &= \sum_{i = 0}^n q^i - \sum_{k = 1}^{n+1} q^{k} \\
            &= (q^0 + \sum_{i = 1}^n q^i) - (\sum_{k = 1}^{n} q^{k} + q^{n+1}) \\
            &= q^0 + \sum_{i = 1}^n q^i - \sum_{k = 1}^{n} q^{k} - q^{n+1} \\
            &= 1 - q^{n+1} \\
        \end{align*}
\end{description}

\subsection{La produttoria}
Si indica con un grande pi greco. E' uguale alla sommatoria ma al posto di
fare la somma fa il prodotto:
\[ \prod_{i \in I}^n a_i \]

\subsubsection{Proprietà}
\begin{itemize}
    \item $\prod_{i \in I} ka_i = k^{\#i}\prod_{i \in I} a_i$
    \item Non vale la dissociativa
\end{itemize} 

\section{Intervalli e intorni}
\subsection{Intervallo}
Per intervallo di estremi $a$ e $b$ si intende \hl{un sottoinsieme di
$\mathbb{R}$ di diversi tipi}:
\begin{itemize}
    \item $(a;b) = \{ x\in\mathbb{R} \mid a<x<b\}$
    \item $[a;b] = \{ x\in\mathbb{R}\mid a \leq x \leq b\}$
    \item $[a;b) = \{ x\in\mathbb{R}\mid a \leq x < b\}$
    \item $(a;b] = \{ x\in\mathbb{R}\mid a < x \leq b\}$
\end{itemize}

Gli intervalli possono essere anche illimitati: $(a; +\infty)$.

\subsection{Intorno}
Preso $x_0 \in \mathbb{R}$, di dice intorno di $x_0$ di raggio $\delta$ 
l'insieme dei valori $x$ tali che:
\[ |x - x_0| < \delta \]

\hl{In generale un intorno è un intervallo $(x_0 - \delta; x_0 + \delta)$, ma un
intervallo non per forza è un intorno.}

\section{Insiemi limitati}
Sia \hl{$E \subseteq \mathbb{R}$}. $E$ è detto \hl{insieme limitato se 
$\exists m,M \in \mathbb{R} \mid  \forall x \in E \; m \leq x \leq M$}. L'insieme 
$E$ è detto superiormente limitato se esiste solo $M$, mentre è detto 
inferiormente limitato se esiste solo $m$.

Un insieme limitato può avere \hl{un massimo e un minimo, però non è detto
che li contenga}. Un esempio di insieme dei questo tipo è $(-1; 1)$. Infatti 
gli elementi, si \hl{continuano ad avvicinare a un valore, ma a causa della
completezza di $\mathbb{R}$, non lo raggiungeranno mai poichè esisterà sempre 
un sepratore} tra l'elemento e il ``bordo''. Ciò sarà ancora più apparente dalla
definizione di massimo e minimo. Per descrivere appieno insiemi come $(-1;1)$
vengono aggiunti i concetti di \hl{maggiorante, minorante, estremo superiore e 
inferiore} che completano quello di massimo e minimo.

\subsection{Massimo di un insieme limitato}
Viene detto \hl{$M$ massimo per un insieme limitato superiormente $E$ se}:
\begin{itemize}
    \item \hl{$\forall x \in E, x \leq M$}
    \item \hl{$M \in E$}
\end{itemize}

\subsection{Minimo di un insieme limitato}
Viene detto \hl{$m$ minimo per un insieme limitato inferiormente $E$ se}:
\begin{itemize}
    \item \hl{$\forall x \in E, x \geq M$}
    \item \hl{$m \in E$}
\end{itemize}

\subsection{Maggiorante di un insieme limitato}
Viene detto \hl{$\bar{M}$ maggiorante di un insieme limitato superiormente $E$}
se \hl{$\forall x \in E, x \geq \bar{M}$}.

Si può notare come \hl{il maggiorante sia una generalizzazione del concetto di 
massimo}. Infatti, \hl{per un insieme superiormente limitato possono esistere 
$\infty$ maggioranti}.

\subsection{Minorante di un insieme limitato}
Viene detto \hl{$\bar{m}$ minorante di un insieme limitato inferiormente $E$}
se \hl{$\forall x \in E, x \leq \bar{m}$}.

Si può notare come \hl{il minorante sia una generalizzazione del concetto di 
minimo}. Infatti, \hl{per un insieme inferiormente limitato possono esistere 
$\infty$ minoranti}.

\subsection{Estremo superiore di un insieme limitato}
Definiamo \hl{$Sup(E)$ estremo superiore di un insieme limitato superiormente $E$
il minimo dei maggioranti}, ossia un numero che:
\begin{itemize}
    \item \hl{$\forall x \in E x \leq a$}
    \item \hl{$a = Min(\mathcal{M})$} 
\end{itemize}
dove $\mathcal{M}$ è l'insieme dei maggioranti di $E$.

\hl{Un insieme limitato superiormente possiede sempre un estremo superiore: esso 
può essere sia interno all'insieme che esterno ad esso}.

\subsection{Estremo inferiore di un insieme limitato}
Definiamo \hl{$Inf(E)$ estremo inferiore di un insieme limitato inferioremente $E$
il massimo dei minoranti}, ossia un numero che:
\begin{itemize}
    \item \hl{$\forall x \in E x \geq a$}
    \item \hl{$a = Max(\mathit{m})$} 
\end{itemize}
dove $\mathit{m}$ è l'insieme dei minoranti di $E$.

\hl{Un insieme limitato inferiormente possiede sempre un estremo inferiore: esso 
può essere sia interno all'insieme che esterno ad esso}.

\subsection{Collegamento tra estremo inferiore (superiore) e l'assioma di completezza}
\hl{Ogni insieme $E \subseteq \mathbb{R}$ limitato inferiormente (superiormente)
ammette estremo inferiore (superiore)}.

\paragraph{Dimostrazione} Prendiamo un \hl{insieme $E$ limitato superiormente}. 
Allora \hl{$E$ ammette maggioranti}. Indichiamo con \hl{$\mathcal{M}$ l'insieme 
di tutti i maggioranti} ($\mathcal{M} = \{x \in R \mid \forall e \in E \; e \leq x\}$). 
L'insieme \hl{$\mathcal{M}$ così definito è limitato inferiormente (tutti gli 
elementi di $E$ sono minoranti di $\mathcal{M}$)}. Definiamo, allora, 
\hl{$\mathcal{N} = \mathbb{R} - \mathcal{M}$} l'insieme di tutti gli elementi 
che non sono maggioranti di $E$. \hl{Osserviamo che}: 
\begin{itemize}
    \item $\mathcal{N} \neq \emptyset$
    \item $\mathcal{M} \cup \mathcal{N} = \mathbb{R}$
    \item $\mathcal{M} \cap \mathcal{N} = \emptyset$
    \item $\forall y \in \mathcal{N} \; \exists \bar{e} \in E \mid \bar{e} > y$,  
        $\forall x \in \mathcal{M} \; \exists \bar{e} \in E \mid x > \bar{e}$ quindi 
        $y < \bar{e} < x$
\end{itemize}
Le osservazioni che abbiamo fatto \hl{non sono altro che le ipotesi dell'assioma
di completezza (vedi {\ref{sec:assioma-completezza}})}. Quindi possiamo affermare
che \hl{$\forall y \in\mathcal{N},\forall x \in\mathcal{M}\;\exists s \mid y \leq s \leq x$}.
\hl{Questo elemento, però, dovrà essere unico in quanto dovrà essere o il minimo di
$\mathcal{M}$ o il massimo di $\mathcal{N}$}. Per dimostrare il teorema dobbiamo 
dimostrare che \hl{$s$ appartiene a $\mathbb{M}$}.

\subparagraph{L'assurdo} \hl{Per assurdo, supponiamo che $s$ appartenga a 
$\mathcal{N}$}. Ciò significa che \hl{$s$ non è un maggiorante} e che 
\hl{$\exists \bar{e} \in E \mid \bar{e} > s$}. Posso, allora, costruire \hl{un 
elemento $s < \frac{s + \bar{e}}{2} < \bar{e}$}. Questo numero è una \hl{contraddizione
perché sarebbe come dire che $\frac{s + \bar{e}}{2} \in \mathcal{N}$ e quindi 
$y \leq s < \frac{s + \bar{e}}{2} < \bar{e} \leq x$}.
Così esisterebbero due elementi separatori, ciò però è un \hl{assurdo perché in
questo caso l'assioma di completezza permette l'esistenza di un solo separatore}.
Allora \hl{$s \in \mathcal{M}$ e di conseguenza $\exists \, Sup(E)$}.



\end{document}
