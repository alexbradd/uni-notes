\section{Intervalli e intorni}
\subsection{Intervallo}
Per intervallo di estremi $a$ e $b$ si intende \hl{un sottoinsieme di
$\mathbb{R}$ di diversi tipi}:
\begin{itemize}
    \item $(a;b) = \{ x\in\mathbb{R} \mid a<x<b\}$
    \item $[a;b] = \{ x\in\mathbb{R}\mid a \leq x \leq b\}$
    \item $[a;b) = \{ x\in\mathbb{R}\mid a \leq x < b\}$
    \item $(a;b] = \{ x\in\mathbb{R}\mid a < x \leq b\}$
\end{itemize}

Gli intervalli possono essere anche illimitati: $(a; +\infty)$.

\subsection{Intorno}
Preso $x_0 \in \mathbb{R}$, di dice intorno di $x_0$ di raggio $\delta$ 
l'insieme dei valori $x$ tali che:
\[ |x - x_0| < \delta \]

\hl{In generale un intorno è un intervallo $(x_0 - \delta; x_0 + \delta)$, ma un
intervallo non per forza è un intorno.}

\section{Insiemi limitati}
Sia \hl{$E \subseteq \mathbb{R}$}. $E$ è detto \hl{insieme limitato se 
$\exists m,M \in \mathbb{R} \mid  \forall x \in E \; m \leq x \leq M$}. L'insieme 
$E$ è detto superiormente limitato se esiste solo $M$, mentre è detto 
inferiormente limitato se esiste solo $m$.

Un insieme limitato può avere \hl{un massimo e un minimo, però non è detto
che li contenga}. Un esempio di insieme dei questo tipo è $(-1; 1)$. Infatti 
gli elementi, si \hl{continuano ad avvicinare a un valore, ma a causa della
completezza di $\mathbb{R}$, non lo raggiungeranno mai poichè esisterà sempre 
un sepratore} tra l'elemento e il ``bordo''. Ciò sarà ancora più apparente dalla
definizione di massimo e minimo. Per descrivere appieno insiemi come $(-1;1)$
vengono aggiunti i concetti di \hl{maggiorante, minorante, estremo superiore e 
inferiore} che completano quello di massimo e minimo.

\subsection{Massimo di un insieme limitato}
Viene detto \hl{$M$ massimo per un insieme limitato superiormente $E$ se}:
\begin{itemize}
    \item \hl{$\forall x \in E, x \leq M$}
    \item \hl{$M \in E$}
\end{itemize}

\subsection{Minimo di un insieme limitato}
Viene detto \hl{$m$ minimo per un insieme limitato inferiormente $E$ se}:
\begin{itemize}
    \item \hl{$\forall x \in E, x \geq M$}
    \item \hl{$m \in E$}
\end{itemize}

\subsection{Maggiorante di un insieme limitato}
Viene detto \hl{$\bar{M}$ maggiorante di un insieme limitato superiormente $E$}
se \hl{$\forall x \in E, x \geq \bar{M}$}.

Si può notare come \hl{il maggiorante sia una generalizzazione del concetto di 
massimo}. Infatti, \hl{per un insieme superiormente limitato possono esistere 
$\infty$ maggioranti}.

\subsection{Minorante di un insieme limitato}
Viene detto \hl{$\bar{m}$ minorante di un insieme limitato inferiormente $E$}
se \hl{$\forall x \in E, x \leq \bar{m}$}.

Si può notare come \hl{il minorante sia una generalizzazione del concetto di 
minimo}. Infatti, \hl{per un insieme inferiormente limitato possono esistere 
$\infty$ minoranti}.

\subsection{Estremo superiore di un insieme limitato}
Definiamo \hl{$Sup(E)$ estremo superiore di un insieme limitato superiormente $E$
il minimo dei maggioranti}, ossia un numero che:
\begin{itemize}
    \item \hl{$\forall x \in E x \leq a$}
    \item \hl{$a = Min(\mathcal{M})$} 
\end{itemize}
dove $\mathcal{M}$ è l'insieme dei maggioranti di $E$.

\hl{Un insieme limitato superiormente possiede sempre un estremo superiore: esso 
può essere sia interno all'insieme che esterno ad esso}.

\subsection{Estremo inferiore di un insieme limitato}
Definiamo \hl{$Inf(E)$ estremo inferiore di un insieme limitato inferioremente $E$
il massimo dei minoranti}, ossia un numero che:
\begin{itemize}
    \item \hl{$\forall x \in E x \geq a$}
    \item \hl{$a = Max(\mathit{m})$} 
\end{itemize}
dove $\mathit{m}$ è l'insieme dei minoranti di $E$.

\hl{Un insieme limitato inferiormente possiede sempre un estremo inferiore: esso 
può essere sia interno all'insieme che esterno ad esso}.

\subsection{Collegamento tra estremo inferiore (superiore) e l'assioma di completezza}
\hl{Ogni insieme $E \subseteq \mathbb{R}$ limitato inferiormente (superiormente)
ammette estremo inferiore (superiore)}.

\paragraph{Dimostrazione} Prendiamo un \hl{insieme $E$ limitato superiormente}. 
Allora \hl{$E$ ammette maggioranti}. Indichiamo con \hl{$\mathcal{M}$ l'insieme 
di tutti i maggioranti} ($\mathcal{M} = \{x \in R \mid \forall e \in E \; e \leq x\}$). 
L'insieme \hl{$\mathcal{M}$ così definito è limitato inferiormente (tutti gli 
elementi di $E$ sono minoranti di $\mathcal{M}$)}. Definiamo, allora, 
\hl{$\mathcal{N} = \mathbb{R} - \mathcal{M}$} l'insieme di tutti gli elementi 
che non sono maggioranti di $E$. \hl{Osserviamo che}: 
\begin{itemize}
    \item $\mathcal{N} \neq \emptyset$
    \item $\mathcal{M} \cup \mathcal{N} = \mathbb{R}$
    \item $\mathcal{M} \cap \mathcal{N} = \emptyset$
    \item $\forall y \in \mathcal{N} \; \exists \bar{e} \in E \mid \bar{e} > y$,  
        $\forall x \in \mathcal{M} \; \exists \bar{e} \in E \mid x > \bar{e}$ quindi 
        $y < \bar{e} < x$
\end{itemize}
Le osservazioni che abbiamo fatto \hl{non sono altro che le ipotesi dell'assioma
di completezza (vedi {\ref{sec:assioma-completezza}})}. Quindi possiamo affermare
che \hl{$\forall y \in\mathcal{N},\forall x \in\mathcal{M}\;\exists s \mid y \leq s \leq x$}.
\hl{Questo elemento, però, dovrà essere unico in quanto dovrà essere o il minimo di
$\mathcal{M}$ o il massimo di $\mathcal{N}$}. Per dimostrare il teorema dobbiamo 
dimostrare che \hl{$s$ appartiene a $\mathbb{M}$}.

\subparagraph{L'assurdo} \hl{Per assurdo, supponiamo che $s$ appartenga a 
$\mathcal{N}$}. Ciò significa che \hl{$s$ non è un maggiorante} e che 
\hl{$\exists \bar{e} \in E \mid \bar{e} > s$}. Posso, allora, costruire \hl{un 
elemento $s < \frac{s + \bar{e}}{2} < \bar{e}$}. Questo numero è una \hl{contraddizione
perché sarebbe come dire che $\frac{s + \bar{e}}{2} \in \mathcal{N}$ e quindi 
$y \leq s < \frac{s + \bar{e}}{2} < \bar{e} \leq x$}.
Così esisterebbero due elementi separatori, ciò però è un \hl{assurdo perché in
questo caso l'assioma di completezza permette l'esistenza di un solo separatore}.
Allora \hl{$s \in \mathcal{M}$ e di conseguenza $\exists \, Sup(E)$}.

