\section{Sommatoria}
Si indica con la \hl{sigma maiuscola}:
\[
    \sum_{i \in I} a_i
\]
Dove:
\begin{itemize}
    \item $I$ è un \hl{insieme finito}. I suoi elementi sono chiamati \hl{indici}
    \item $(a_i), i \in I$ è una \hl{famiglia di numeri che dipendono da $i$}
\end{itemize}

\paragraph{Alcune sommatorie famose}
\begin{description}
    \item[Formula di Gauss] $\sum_{i=1}^n (i) = \frac{n\cdot(n-1)}{2}$
    \item[Somma di una progressione geometrica]
        \begin{align*}
            \sum_{i = 0}^n q^i &= \frac{1-q^{n+1}}{1-q} \\
            &= n+1 \text{ per } q = 1
        \end{align*}
        Dimostrazione:
        \begin{align*}
            \text{Tesi: } (1-q)\sum_{i = 0}^n q^i &= 1-q^{n+1} \\
            (1-q)\sum_{i = 0}^n q^i &= \sum_{i = 0}^n q^i - q \sum_{i = 0}^n q^i \\
            &= \sum_{i = 0}^n q^i - \sum_{i = 0}^n q^{i+1} \text{ prendiamo } k = i + 1 \\
            &= \sum_{i = 0}^n q^i - \sum_{k = 1}^{n+1} q^{k} \\
            &= (q^0 + \sum_{i = 1}^n q^i) - (\sum_{k = 1}^{n} q^{k} + q^{n+1}) \\
            &= q^0 + \sum_{i = 1}^n q^i - \sum_{k = 1}^{n} q^{k} - q^{n+1} \\
            &= 1 - q^{n+1} \\
        \end{align*}
\end{description}

\paragraph{Le proprietà della sommatoria} 
\begin{itemize}
    \item La sommatoria è \hl{un operatore lineare}
    \item \hl{l'indice è muto}: non importa il nome dell'indice 
    \item \hl{traslando gli indici, la sommatoria non cambia}: è importante 
        che il numero di elementi sia uguale
    \item \hl{si definiscono sommatorie anche su due o più famiglie di indici}: 
        $\sum_{i \in I, j \in J} a_{ij} = \sum_{i \in I}\sum_{j \in J} a_{ij}$
    \item \hl{vale la proprietà dissociativa}:
        $\sum_{i \in I} (a_i + b_i) = \sum_{i \in I} (a_i) + \sum_{i \in I} (b_i)$
    \item \hl{le costanti possono essere portate fuori}:
        $\sum_{i \in I} Ka_i = K\cdot\sum_{i \in I} a_i$
    \item \hl{può essere scomposta in sommatorie più piccole}:
        $\sum_{i=1}^n a_i = \sum_{i=1}^k a_i + \sum_{i=k+1}^n a_i$
    \item \hl{riflessione degli indici}:
        $\sum_{i=0}^n = \sum_{i=0}^n a_{n-i}$
\end{itemize}

\section{La produttoria}
Si indica con un grande pi greco. E' uguale alla sommatoria ma al posto di
fare la somma fa il prodotto.

\paragraph{Proprietà}
\begin{itemize}
    \item $\prod_{i \in I} ka_i = k^{\#i}\prod_{i \in I} a_i$
    \item Non vale la dissociativa
\end{itemize} 
