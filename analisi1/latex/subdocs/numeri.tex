\section{Insiemi numerici}
\subsection{Numeri naturali}
Sono i numeri \hl{interi positivi incluso lo 0}. Può \hl{essere costruito a 
partire da un solo numero (lo 0)} basta aggiungendo un'unità ogni volta.
\[ \mathbb{N} = \{0, 1, 2, 3, \dots \} \]

\subsubsection{Proprietà} 
\hl{Contiene sempre il successore di ogni suo elemento 
(principio di induzione)}. Gode della \hl{relazione d'ordine $\leq$, il che lo 
rende un insieme ordinato}. $\mathbb{N}$, come tutti i suoi sottoinsiemi, 
\hl{godono del principio del minimo intero} che lo rende, insieme ai suoi 
sottoinsiemi, \hl{un insieme \textit{ben} ordinato}.

\subsubsection{Operazioni definite} 
In $\mathbb{N}$ sono definite somma e prodotto: in questo modo:\[
    \begin{array}{cccc}
        +: &\mathbb{N} \times \mathbb{N} &\to &\mathbb{N} \\
        &somma(n1, n2) &\mapsto & n_1 + n_2 \in \mathbb{N}
    \end{array} \quad
    \begin{array}{cccc}
        \cdot: &\mathbb{N} \times \mathbb{N} &\to &\mathbb{N} \\
        &prodotto(n1, n2) &\mapsto & n_1 \cdot n_2 \in \mathbb{N}
    \end{array}
\]

\paragraph{Proprietà delle operazioni}
\begin{description}
    \item[Commutativa] $n_1 + n_2 = n_2 + n_1$
    \item[Associativa] $n_1 + (n_2 + n_3) = (n_1 + n_2) + n_3$
    \item[Distributiva] $n_1 \cdot (n_2 + n_3) = n_1 \cdot n_2 + n_1 \cdot n_3$
\end{description}

\subsubsection{Principio del minimo intero} 
\hl{Ogni sottoinsieme di $\mathbb{N}$ ha un elemento minimo (più piccolo di 
tutti gli altri)}.

\subsubsection{Il principio di induzione} 
Sia \hl{$S \subseteq \mathbb{N}$ un sottoinsieme tale che 
$0 \in S$ e $\forall n \in S \implies n+1 \in S$}. Allora 
\hl{S coincide con $\mathbb{N}$}.

\paragraph{Il principio di induzione nella logica} Il principio di induzione 
può essere usato per dimostrare teoremi in $\mathbb{N}$. Enunciamolo in questo 
modo: sia \hl{$P(n)$ un predicato che dipende da $n \in \mathbb{N}$} tale che
\hl{$P(n_0)$ sia vero e che $\forall n \in \mathbb{N} P(n) \implies P(n+1)$}. Il
predicato \hl{sarà vero per tutti gli $n \geq n_0$}.

\subsubsection{Fattoriale} 
Preso $n \in N$, il fattoriale di $n$ sarà
$n! = 1 \cdot 2 \cdot 3 \cdot \ldots \cdot (n-1) \cdot n$. Una eccezione è lo 0:
il fattoriale di 0 è $0! = 1$. Il fattoriale è un numero definito che può essere
definito induttivamente: $n! = n(n-1)!$.

\subsubsection{Coefficiente binomiale} 
Già incontrati nella probabilità:
\[ \binom{n}{n} = \frac{n!}{k!(n-k)!} \]
con $n \in \mathbb{N}, 0 \leq k \leq n$. Convenzionalmente $\binom{0}{0} = 1$. 
Il coefficiente binomiale viene usato nel binomio di Newton.

\subsubsection{Binomio di Newton} 
Il binomio di Newton ci permette di calcolare l'elevamento a qualsiasi potenza 
di un binomio:
\[ (a+b)^n = \sum_{k=0}^n \binom{n}{k} a^{n-k} b^k \]
La formula è dimostrabile per induzione (se sei proprio interessato, vedi gli
appunti a penna).

\subsection{Numeri interi relativi}
\'E l'insieme $\mathbb{Z}$. \hl{Non esiste un minimo, di conseguenza non valgono 
il principio del minimo intero e il principio di induzione}. \'E \hl{definita la
relazione d'ordine $\leq$, quindi è un insieme ordinato} ma a causa della
mancata validità dei due principi nominati precedentemente, \hl{non è un insieme
ben ordinato}. $\mathbb{Z}$ è, inoltre, più grande di $\mathbb{N}$: $ N \subset Z$.

\subsubsection{Costruzione} 
Per costruire il numeri relativi, \hl{definiamo una relazione di equivalenza 
$\sim$ in $\mathbb{N} \times \mathbb{N}$ tale che}:
\[ (a,b) \sim (h,k) \iff a+k=b+h \]
Questa relazione di equivalenza \hl{ci permette di descrivere tutti i numeri negativi
che sono la differenza dei numeri $a$ e $b$ o $h$ e $k$}: per esempio \hl{$-1$ è la 
classe di equivalenza $[(2,3)]_{\sim}$}. $\mathbb{Z}$ viene, quindi, definito come
$\mathbb{Z} = (\mathbb{N} \times \mathbb{N}) / \sim$

\paragraph{Dimostrazione che $\sim$ è una relazione di equivalenza} Per
dimostrare che $\sim$ è una relazione di equivalenza, verifichiamo che soddisfi
i requisiti:
\begin{itemize}
    \item è riflessiva: $(m,n) \sim (m,n) \implies m+n = n+m$
    \item è simmetrica: $(a,b) \sim (c,d) = (c,d) \sim (a,b)$
    \item è transitiva: $(a,b) \sim (c,d), (c,d) \sim (e,f) \implies (a,b)\sim(e,f)$
        Infatti: 
        \begin{align*}
            a+d=b+c, &\quad c+f=d+e \\
            a-b=c-d, &\quad c-d=e-f \\
            a-b&=e-f \\
            a+f&=b+e
        \end{align*}
\end{itemize}

\subsubsection{Operazioni definite} 
Le operazioni sono \hl{le stesse di $\mathbb{N}$ ma aggiornate}:
\[
    \begin{array}{cccc}
        +: &\mathbb{Z} \times \mathbb{Z} &\to &\mathbb{Z} \\
        &((a, b)_{\sim}, (h,k)_{\sim}) &\mapsto & (a+h, b+k)_{\sim}
    \end{array} \quad
    \begin{array}{cccc}
        \cdot: &\mathbb{Z} \times \mathbb{Z} &\to &\mathbb{Z} \\
        &((a, b)_{\sim}, (h,k)_{\sim}) &\mapsto & (ah + bk, bh + ak)_{\sim}
    \end{array}
\]

\paragraph{Proprietà delle operazioni} Mantnengono le stesse proprietà che 
avevano in $\mathbb{N}$.

\subsection{Numeri razionali}
\'E l'insieme $\mathbb{Q}$. \hl{Non esiste un minimo, di conseguenza non valgono 
il principio del minimo intero e il principio di induzione}. \'E \hl{definita la
relazione d'ordine $\leq$, quindi è un insieme ordinato} ma a causa della
mancata validità dei due principi nominati precedentemente, \hl{non è un insieme
ben ordinato}.

\subsubsection{Costruzione}
Per costruire il numeri razionali, \hl{definiamo una relazione di equivalenza 
$\approx$ in $\mathbb{Z} \times (\mathbb{Z} - \{0\})$ tale che}:
\[ (a,b) \approx (h,k) \iff ak=bh \]
Questa relazione di equivalenza \hl{ci permette di descrivere tutti i numeri razionali
che sono divisione dei numeri $a$ e $b$ o $h$ e $k$}: per esempio \hl{$\sfrac{2}{3}$
è la classe di equivalenza $[(2,3)]_{\approx}$}. $\mathbb{Q}$ viene, quindi, 
definito come $\mathbb{Z} = (\mathbb{Z} \times (\mathbb{Z} - \{0\}) / \approx$

\paragraph{Dimostrazione che $\approx$ è una relazione di equivalenza} Per
dimostrare che $\approx$ è una relazione di equivalenza, verifichiamo che soddisfi
i requisiti:
\begin{itemize}
    \item è riflessiva: $(m,n) \approx (m,n) \implies mn = nm$
    \item è simmetrica: $(a,b) \approx (c,d) = (c,d) \approx (a,b)$
    \item è transitiva: $(a,b) \approx (c,d), (c,d) \approx (e,f) \implies (a,b)\approx(e,f)$
        Infatti: 
        \begin{align*}
            ad=bc, &\quad cf=de \\
            \frac{a}{b}=\frac{c}{d}, &\quad \frac{c}{d}=\frac{e}{f} \\
            \frac{a}{b}&=\frac{e}{f} \\
            af&=be
        \end{align*}
\end{itemize}

\subsubsection{Operazioni definite} 
Le operazioni sono \hl{le stesse che sono definite in $\mathbb{Z}$ ma aggiornate}:
\[
    \begin{array}{cccc}
        +: &\mathbb{Q} \times \mathbb{Q} &\to &\mathbb{Q} \\
        &((a, b)_{\approx}, (h,k)_{\approx}) &\mapsto & (ak+bh, b+k)_{\sim}
    \end{array} \quad
    \begin{array}{cccc}
        \cdot: &\mathbb{Q} \times \mathbb{Q} &\to &\mathbb{Q} \\
        &((a, b)_{\approx}, (h,k)_{\approx}) &\mapsto & (ah, bk)_{\approx}
    \end{array}
\]

\paragraph{Proprietà delle operazioni} Mantengono le stesse proprietà che 
avevano in $\mathbb{Z}$.

\subsubsection{La rappresentazione decimale}
La rappresentazione decimale di un numero non è nient'altro che un \hl{allineamento
di cifre}. Le rappresentazioni decimali che si trovano \hl{nei razionali sono 
limitate, illimitate periodiche}. Esistono anche rappresentazioni \hl{illimitate,
ma non sono contenute in $\mathbb{Q}$}.

\'E costituita da una \hl{parte intera (necessariamente finita)} e una \hl{parte
decimale che può essere finita o illimitata} (si ricorda che in $\mathbb{Q}$ solo
illimitati periodici). Può essere scritta come:
\[ x = \pm \sum_{j=0}^k c_j \cdot 10^j + \sum_{l=0}^m d_l 10^{-l} \]
Dove la prima sommatoria rappresenta la parte intera e la seconda la parte 
decimale.

\subsection{I numeri reali}
L'insieme dei numeri reali contiente \hl{qualsiasi rappresentazione decimale
possibile, limitata o illimitata}. Di conseguenza, $\mathbb{R}$ contiene tutti
gli insiemi visti fino ad ora. Nell'insieme dei reali è \hl{definita la relazione
d'ordine $\leq$, rendolo un insieme ordinato}. Inoltre, vale anche \hl{l'assioma
di completezza, che rende $\mathbb{R}$ un insieme ordinato e completo}.

\subsubsection{Operazioni definite}
Le operazioni definite sono \hl{sempre le stesse trovate negli insiemi precedenti}:
\[
    \begin{array}{cccc}
        +: &\mathbb{R} \times \mathbb{R} &\to &\mathbb{R} \\
        &somma(a,b) &\mapsto & a + b \in \mathbb{R}
    \end{array} \quad
    \begin{array}{cccc}
        \cdot: &\mathbb{R} \times \mathbb{R} &\to &\mathbb{R} \\
        &prodotto(a,b) &\mapsto & a \cdot b \in \mathbb{R}
    \end{array}
\]

\paragraph{Proprietà delle operazioni} Mantengono le stesse proprietà che 
avevano in $\mathbb{Q}$.

\subsubsection{Assioma di completezza}\label{sec:assioma-completezza}
Siano \hl{$A, B \subseteq R$ tali che}:
\begin{itemize}
    \item $A,B \neq \emptyset$
    \item $A \cap B = \emptyset$ 
    \item $A \cup B = R$
    \item $\forall a \in A, \forall b \in B \; a < b$ 
\end{itemize}
allora esiste un unico numero reale tale che 
\hl{$\forall a \in A, \forall b \in B \; a \leq s \leq b$}. $s$ è detto elemento 
\hl{separatore}. 

\subsection{Numeri complessi}
\'E l'insieme che \hl{completa i numeri reali}: \hl{ci permettono di risolvere le
equazioni polinomiali che non riuscivamo nei reali (chiusura algebrica)}. 

\subsubsection{Costruzione}
\'E un \hl{insieme di coppie ordinate di numeri reali} appartenenti a 
$\mathbb{R} \times \mathbb{R}$.

\subsubsection{Operazioni}
Le operazioni sono \hl{sempre le stesse che in $\mathbb{R}$ ma aggiornate}:
\begin{align*}
    \begin{array}{cccc}
        +: &\mathbb{C} \times \mathbb{C} &\to &\mathbb{C} \\
        &somma((a,b), (c,d)) &\mapsto & (a+c, b+d) \in \mathbb{R}
    \end{array} \\
    \begin{array}{cccc}
        \cdot: &\mathbb{C} \times \mathbb{C} &\to &\mathbb{C} \\
        &prodotto((a,b), (c,d)) &\mapsto & (ac-bd, ad + bc) \in \mathbb{R}
    \end{array}
\end{align*} 
