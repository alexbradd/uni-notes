\section{Insiemi}
\hl{Un insieme è una collezione di oggetti}. Tutta la matematica si basa
sulla teoria assiomatica degli insiemi. \hl{Un insieme A si può indicare
per elencazione ($A = \{a_1, \dots, a_n\}$) o con una condizione
($A = \{x\mid \textit{condizione}\}$)}. La \hl{cardinalità di $A$ è il numero di
oggetti: $|A| = n$}. La cardinalità dell'insieme vuoto è 0.

\paragraph{Esempi} $\mathbb{N} = \{0, 1, 2,\dots\}$,
$\mathbb{Q} = \{q = \frac{m}{n} \mid m,n \in \mathbb{Z}, n \neq 0\}$,
$\mathbb{R} = \{\text{x numeri decimali}\}$.

Un \hl{insieme particolare è l'insieme con nessun elemento detto vuoto, indicato
con $\emptyset$}. Un altro insieme particolare è \hl{l'insieme di tutti gli tutto
detto insieme universo $U$}.

\subsection{Sottoinsiemi}
Un insieme può essere sottoinsieme di un altro, ossia contenere una parte degli
elementi dell'insieme più grande. Formalizzando si può dire che:
\[
    A \subset B \implies \forall a \in A, a \in B
\]

\subsection{Operazioni}
Le operazioni più usate sono:
\begin{description}
    \item[Unione] $A \cup B = \{x \mid x \in A \vee x \in B\}$
    \item[Intersezione] $A \cap B = \{x \mid x \in A \wedge x \in B\}$
    \item[Complementare] $A^C = \bar{A} = \{x \in U \mid x \notin A\}$
    \item[Differenza] $A - B = \{x \mid x \in A \wedge x \notin B\}$
        Si può anche trovare indicata con $ \setminus $
    \item[Prodotto cartesiano] $A \times B = \{(a,b) \mid a \in A, b \in B \}$
        Le coppie $(a,b)$ sono anche dette \hl{coppie} (m-uple per $m$ elementi)
\end{description}

\section{Relazioni}
\hl{Una relazione è un sottoinsieme del prodotto cartesiano tra due insiemi}.

Per indicare che due elementi $(a_i, b_j)$ sono legati da una relazione
$R$ usiamo \hl{$a_i \sim_R b_j$}. Per rappresentare le relazioni si possono usare i
diagrammi di Venn (le patate) con le frecce che collegano i vari elementi tra di
loro.

\paragraph{Esempio} Presi $A = \{a_1, a_2\}, B = \{b_1, b_2\}$, calcoliamo il
loro prodotto cartesiano e otterremo 16 possibili sottoinsiemi:
\begin{align*}
    R_0 &= \emptyset \\
    R_1 &= \{(a_1, b_1)\}, \dots, R_4 \\
    R_5 &= \{(a_1, b_1), (a_1, b_2)\}, \dots, R_{10} \\
    R_{11} &= \{(a_1, b_1), (a_1, b_2), (a_2, b_1)\}, \dots, R_{14} \\
    R_{15} &= A \times B
\end{align*}

\subsection{Relazioni particolari}
\paragraph{Relazione d'ordine} Prendiamo una \hl{relazione
$R \subseteq A \times A$, essa è d'ordine se}:
\begin{itemize}
    \item \hl{è riflessiva}: $(a,a) \in R \forall a \in R$
    \item \hl{è antisimmetrica}: $(a,b),(b,a) \in R \implies a=b$
    \item \hl{è transitiva}: $(a,b), (b,c) \in R \implies (a,c) \in R$
\end{itemize}

\subparagraph{Insieme totalmente e parzialmente ordinato} Siano $A$ un insieme ed
$R$ una relazione d’ordine su $A$. \hl{Se per ogni $a1 , a2 \in A$ vale
$(a1, a2) \in R$ oppure $(a2 , a1 ) \in R$, $R$ si dice relazione d’ordine
totale e la coppia $(A, R)$ si dice insieme totalmente ordinato. In caso
contrario si dice che $R$ è una relazione d’ordine parziale e la coppia $(A, R)$
si dice insieme parzialmente ordinato}.

\paragraph{Relazione di equivalenza} Prendiamo una \hl{relazione
$R \subseteq A \times A$, essa è di equivalenza se}:
\begin{itemize}
    \item \hl{è riflessiva}: $(a,a) \in R \forall a \in R$
    \item \hl{è simmetrica}: $(a,b) \in R \implies (b,a) \in R$
    \item \hl{è transitiva}: $(a,b), (b,c) \in R \implies (a,c) \in R$
\end{itemize}
Una modo di vedere la relazione di equivalenza è come \hl{generalizzazione
dell'uguaglianza}.

\subparagraph{Classe di equivalenza} Data una relazione di equivalenza $R$, preso
un elemento $a$, la \hl{classe di equivalenza di $a$ sono tutti gli elementi
equivalenti equivalenti ad $a$, ossia:}
    \[ {[a]}_R = \{ b \in A \mid a \sim_R a \} \]
La classe di equivalenza è in sostanza l'insieme di tutti gli elementi
equivalenti tra di loro.

\hl{Teorema}: Ogni elemento $a \in A$ appartiene a una sola classe di equivalenza
(dimostrazione nella dispensa, teorema 2.38).
\hl{Teorema}: Un insieme $A$ sul quale agisce una relazione di equivalenza $R$
è l'unione disgiunta delle sue classi di equivalenza.

\subparagraph{Insieme quoziente} \hl{L'insieme quoziente $A/R$} di $A$ rispetto
a una relazione di equivalenza $R$ è \hl{l'insieme di tutte le classi di
equivalenza}.

\section{Funzioni}
\hl{Le funzioni sono speciali relazioni che associano a ogni elemento del
primo insieme un solo elemento del secondo}. Una funzione in genere si indica
con la lettera minuscola e usa questa notazione: \[ f: A \to B \]
\hl{L'insieme $A$ è detto dominio, $B$ il codominio. L'insieme di tutte le
possibili funzioni che vanno da $A$ a $B$ si indica con $B^A$.}

Preso \hl{$a \in A, b = f(a)$ sarà la sua immagine}. La \hl{controimmagine
di $b$ è l'elemento tale che \\ $f^{-1}(b) = \{ a \in A \mid f(a) = b \}$}

\hl{L'insieme di tutte le immagini è detto insieme immagine} e si indica con
$Im(f)$.

Le funzioni sono trattate più nel dettaglio nell'omonimo capitolo degli appunti
di Analisi 1

\paragraph{Funzione particolare} \hl{La funzione
$A \times A = \Delta A = Id(A) = \{(a,a) \mid a \in A\}$ è detta funzione identità
o insieme diagonale}.

\paragraph{Iniettività} Una funzione è detta \hl{iniettiva se
$\forall a,b \in A, a \neq b \implies f(a) \neq f(b)$}.

\paragraph{Suriettività} Una funzione è detta \hl{suriettiva se
$\forall b \in B, \exists a \in A \mid f(a) = b$}.

\paragraph{Funzione biunivoca} Se una funzione è \hl{sia iniettiva che
suriettiva è detta biunivoca}. Se una funzione è biunivoca può essere invertita
ottenendo $f^{-1}: B \to A$.

\paragraph{Composizione di funzioni} Date due funzioni $f: A \to B, g: B \to C$,
la composizione $g \circ f$ delle due è una nuova funzione tale che
$g \circ f: A \to C$. Ciò equivale a dire che $(g \circ f)(a) = g(f(a))$

\section{Operazioni}
Le operazioni sono delle \hl{speciali funzioni: dati $n+1$ insiemi
$A_1, \dots, A_{n+1}$ non vuoti, una operazione n-aria $\ast$ è una funzione
che}:\[
    \begin{array}{cccc}
        \ast: &A_1 \times \cdots \times A_{n} &\to &A_{n+1} \\
        &(a_1, \dots, a_n) &\mapsto & \ast (a_1, \dots, a_n)
    \end{array}
\]
Se \hl{$A_1 = \cdots = A_{n+1}$ allora l'operazione è detta interna}, altrimenti
è detta esterna. Se \hl{$n = 2$ allora l'operazione è detta binaria} e si può
indicare con $a_1 \ast a_2$.

\paragraph{Esempi}La somma + un'operazione binaria interna a $\mathbb{N}$
\[
    \begin{array}{cccc}
       +: &\mathbb{N} \times \mathbb{N} &\to &\mathbb{N} \\
        &(n1, n2) &\mapsto &n3 = n1 + n2
    \end{array}
\]
La differenza è sempre un'operazione binaria, ma esterna ad $\mathbb{N}$
\[
    \begin{array}{cccc}
        -: &\mathbb{N} \times \mathbb{N} &\to &\mathbb{Z} \\
        &(n1, n2) &\mapsto &n3 = n1 - n2
    \end{array}
\]

Le varie operazioni possono essere rappresentate in tabelle che indicano tutti i
possibili casi. Ad esempio, esistono $2^4 = 16$ diverse operazioni binarie
interne ($\ast: A \times A \to A$) ad $A = \{a_1, a_2\}$.

\paragraph{Proprietà delle operazioni} Le operazioni possono godere di alcune
proprietà:
\begin{description}
    \item[Elemento neutro] $a \ast e = a$
    \item[Inverso] $a \ast a^{-1} = e$
    \item[Proprietà commutativa] $a \ast b = b \ast a$
    \item[Proprietà assocativa] $a \ast (b \ast c) = (a \ast b) \ast c$
    \item[Proprietà distributiva] Lega due operazioni:
        $a \cdot (b \ast c) = (a \cdot b) \ast (a \cdot c)$
\end{description}

\section{Polinomi}
Un polinomio $P(x)$ è una \hl{particolare funzione della forma}:
\[
    P(x) = \sum_{i=0}^{n} a_i x^i \text{ con } n \in \mathbb{N}
\]
Dove \hl{$(a_1, \ldots, a_n)$ (i coefficienti) appartengono a un campo $K^{n+1}$}.
\hl{L'insieme di tutti i possibili coefficienti si indica con $K[x]$}.
Un polinomio nelle m variabili $x_1, \dots ,x_m$ è definito induttivamente come
l’espressione:
\[
    P(x_1, \ldots, x_m) = \sum_{i=0}^n Q_i(x_1, \ldots, x_{m-1})x_m^i
\]
dove $Q_1, \ldots, Q_n$ sono polinomi nelle prime $m - 1$ variabili. L’insieme
di tutti i polinomi di questo tipo si indica con $K[x_1, \ldots, x_m]$.

Se il campo \hl{$K$ coincide con il campo dei reali ($(\mathbb{R}, +, \times)$)
allora $K[x] = \mathbb{R}[x]$} e sarà l'insieme di tutti i possibili polinomi
con variabile reale.

Un polinomio è generalmente scritto come somma di monomi.

\paragraph{Il grado di un polinomio} Il grado di un polinomio $P(x)$ è il
\hl{massimo grado dei suoi monomi con grado diverso da 0}. Il polinomio nullo
ha per definizione grado indeterminato.

\subsection{Divisione tra polinomi}
Data la coppia \hl{$(A, B) \in K[x] \times K[x], B \neq 0$, esiste una sola coppia
$(Q, R) \in K[x] \times K[x]$ tale che \\ $A = QB + R$ per la quale $grado(R) < grado(Q)$
o $grado(R) = 0$}. $Q$ e $R$ sono rispettivamente quoziente e resto della divisione
di $A$ e $B$.

\paragraph{Molteplicità algebrica} Dati \hl{$P \in K[x], r \in \mathbb{N}$} esiste
un valore \hl{$m < grado(P)$} tale che \hl{${(x-r)}^m$ divida $P(x)$}. Tale valore è
detto \hl{molteplicità algebrica di $r$ rispetto a $P$. La $r$ sarà la radice} del
polinomio. \hl{Se la molteplicità algebrica di $r$ è $1$, $r$ è una radice semplice.}

\paragraph{Chiusura algebrica} Le \hl{radici di un polinomio $P \in K[x]$} di grado
$n$ rispettano la regola \\ \hl{$m_1 + \cdots + m_i \leq n$} dove \hl{$m_i$ è la
molteplicità algebrica di $r_i$} con $i = 1, \ldots, k$. \hl{Per ogni campo $K$
esisterà un altro campo $\bar{K}$ che lo contiene} tale che ogni polinomio
appartenente ad esso \hl{abbia le radici che soddisfino $m_1 + \cdots + m_i = n$}.
Tale campo è detto \hl{chiusura algebrica di $K$}. Se \hl{$K$ e la sua chiusura
coincidono, $K$ si dice algebricamente completo}.

\hl{Il campo dei $\mathbb{C}$ è algebricamente chiuso, è la chiusura algebrica di
$\mathbb{R}$ e contiene la chiusura algebrica di $\mathbb{Q}$}. 
